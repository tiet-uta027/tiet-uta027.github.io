% Created 2025-04-02 Wed 07:43
% Intended LaTeX compiler: pdflatex
\documentclass[11pt]{article}
\usepackage[utf8x]{inputenc}
\usepackage[T1]{fontenc}
\usepackage{graphicx}
\usepackage{longtable}
\usepackage{wrapfig}
\usepackage{rotating}
\usepackage[normalem]{ulem}
\usepackage{amsmath}
\usepackage{amssymb}
\usepackage{capt-of}
\usepackage{hyperref}
\usepackage{minted}
\usepackage{parskip}
\usepackage{svg}
\svgpath{{../../assets/icons/}}
\newcommand{\bvricon}[1]{\includesvg[scale=0.7]{#1}}
\author{Raghav B. Venkataramaiyer}
\date{Mar '25}
\title{Assignment 05 : Neural Regression (Iris Dataset)\\\medskip
\large UTA027 : Artificial Intelligence}
\hypersetup{
 pdfauthor={Raghav B. Venkataramaiyer},
 pdftitle={Assignment 05 : Neural Regression (Iris Dataset)},
 pdfkeywords={},
 pdfsubject={},
 pdfcreator={Emacs 29.4 (Org mode 9.6.24)}, 
 pdflang={English}}
\begin{document}

\maketitle
\textbf{Assignment 05: Multi-Layer Neural Network on the Iris
Dataset Using PyTorch} \\[0pt]
(07-Apr to 18-Apr)

\section{Objective}
\label{sec:org1f3a00b}
The goal of this assignment is to implement a
multi-layer neural network (MLP) using PyTorch to
classify the Iris dataset.
\section{Question}
\label{sec:org2f47a12}
Design and implement a multi-layer neural network using
PyTorch to classify the Iris dataset. Your
implementation should follow these steps:
\subsection{Dataset Preparation}
\label{sec:orgcb1faac}
\begin{itemize}
\item Load the Iris dataset using
sklearn.datasets.load\textsubscript{iris}.
\item Convert the dataset into PyTorch tensors.
\item Split the dataset into training and test sets (e.g.,
80\% training, 20\% testing).
\item Normalize the feature values.
\end{itemize}
\subsection{Build the Neural Network Model}
\label{sec:orge2a71b5}
\begin{itemize}
\item Implement an MLP with PyTorch using
\texttt{torch.nn.Module}.
\item The model should have:
\begin{itemize}
\item An input layer with 4 neurons (one for each
feature).
\item At least one hidden layer with ReLU activation.
\item An output layer with 3 neurons (one for each class)
and softmax activation.
\end{itemize}
\end{itemize}
\subsection{Train the Model}
\label{sec:org164fcbb}
\begin{itemize}
\item Define the loss function (CrossEntropyLoss).
\item Choose an optimizer (e.g., Adam or SGD).
\item Train the model for a fixed number of epochs (e.g.,
100 epochs).
\item Track the loss during training.
\end{itemize}
\subsection{Evaluate the Model}
\label{sec:org8a2ff8a}
Compute accuracy on the test set.  Generate a confusion
matrix to visualize performance.

\section{Theory}
\label{sec:orgb13527f}
\begin{itemize}
\item \href{https://docs.google.com/presentation/d/1Y0N7jhqgCFR6K1e48iIxqRxBkzKXEe27QUDyQ9\_DGLc/edit?usp=sharing}{\bvricon{simple/googleslides}
  [Click here] Neuron
and it application in Regression/ Classification.}
\end{itemize}

\section{Boilerplate Code}
\label{sec:orgdce8c55}
\begin{itemize}
\item \href{https://gist.github.com/bvraghav/3e28f1f44eaf0dc74842b3e2395e86bd}{\bvricon{simple/github} +
\bvricon{simple/googlecolab}
  [Click here] How to create a neural network
module.}
\item \href{https://gist.github.com/bvraghav/fdfeebb9c73f5d27aa98e74409adb38b}{\bvricon{simple/github} +
\bvricon{simple/googlecolab}
  [Click here] How to create a pytorch dataset from
tensors.}
\end{itemize}
\section{Evaluation Criterion}
\label{sec:orga97672c}

This assignment shall be implemented by students and
evaluated by the instructor in lab.
\end{document}

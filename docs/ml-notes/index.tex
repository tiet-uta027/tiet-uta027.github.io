% Created 2025-02-13 Thu 06:45
% Intended LaTeX compiler: pdflatex
\documentclass[11pt]{article}
\usepackage[utf8x]{inputenc}
\usepackage[T1]{fontenc}
\usepackage{graphicx}
\usepackage{longtable}
\usepackage{wrapfig}
\usepackage{rotating}
\usepackage[normalem]{ulem}
\usepackage{amsmath}
\usepackage{amssymb}
\usepackage{capt-of}
\usepackage{hyperref}
\usepackage{minted}
\usepackage{parskip}
\author{Raghav B. Venkataramaiyer}
\date{Feb '25}
\title{Machine Learning Notes\\\medskip
\large Prerequisites, Regression and Classification}
\hypersetup{
 pdfauthor={Raghav B. Venkataramaiyer},
 pdftitle={Machine Learning Notes},
 pdfkeywords={},
 pdfsubject={},
 pdfcreator={Emacs 29.4 (Org mode 9.6.24)}, 
 pdflang={English}}
\begin{document}

\maketitle

\section{Setup}
\label{sec:org81bc0b5}

\begin{enumerate}
\item Given is a set of paired observations \(\mathcal{D}\)
(aka evidence), where targets \(y\in\mathbb{R}\) are
paired with (\(d\) dimensional) features
\(\mathbf{x}\in\mathbb{R}^{d}\).
\item We propose a mathematical model (typically a
“family” of functions)
\(\mathcal{F}_{\boldsymbol{\theta}} : \mathbb{R}^{d}
   \to \mathbb{R}\) parameterised by
\(\boldsymbol{\theta}\)
\item So that \(y\approx
   \mathcal{F}_{\boldsymbol{\theta}_{*}} (\mathbf{x})\).
Here \(y\) are referred to as targets,
\(\mathcal{F}_{\boldsymbol{\theta}} (\mathbf{x})\) are
referred to as predictions; so that predictions
approximate the targets, under optimal set of learnt
parameters, \({\boldsymbol{\theta}_{*}}\).
\item We express this formally as: \\[0pt]
Find \({\boldsymbol{\theta} =
   {\boldsymbol{\theta}_{*}}}\) in order to

\begin{align*}
  \underset{\boldsymbol{\theta}} {\text{minimise}}
  \quad
  &\underset{y,\mathbf{x}\sim\mathcal{D}}{\mathbb{E}}
    \left[ \Delta(y, \mathcal{F}_{\boldsymbol{\theta}}
    (\mathbf{x})) \right]
\end{align*}

where, \(\Delta\) is the notion of distance between
predictions \(\mathcal{F}_{\boldsymbol{\theta}}
   (\mathbf{x})\) and targets \(y\).
\end{enumerate}

\section{Linear Regression}
\label{sec:org4937dfb}

\subsection{In 2D}
\label{sec:orgb7b6340}
\begin{align*}
  y \approx \mathcal{F}_{w,b}(x)
  &= wx+b \\
  \Delta\left(y, \mathcal{F}_{w,b}(x)\right)
  &= \frac12 \left(y - \mathcal{F}_{w,b}(x) \right)^2
\end{align*}

The objective is to find \(w=w_*\), \(b=b_*\) in order to

\begin{align*}
  \underset{w,b}{\text{minimise}}
  &\quad \underset{y,x\sim\mathcal{D}}{\mathbb{E}}
    \left[ \frac12 \left(y - \mathcal{F}_{w,b}(x)
    \right)^2 \right]
\end{align*}

The analytical solution yields,

\begin{align*}
  w_* &= \frac{\mathrm{coVar}(x,y)}{\mathrm{Var}(x)} \\
  b_* &= \mathbb{E}[y]-w_*\mathbb{E}[x] \\
  \mathrm{coVar}(x,y) &= \mathbb{E}[xy] -
                        \mathbb{E}[x]\mathbb{E}[y] \\ 
  \mathrm{Var}(x) &= \mathbb{E}[x^2]-\mathbb{E}^2[x]
\end{align*}

\subsection{In Higher Dimensions}
\label{sec:orgf3a4778}

\begin{align*}
  y \approx \mathcal{F}_{\mathbf{w}}(\mathbf{x})
  &= \mathbf{w}^{\top}\mathbf{x} =
    \mathbf{x}^{\top}\mathbf{w} \\
  &= w_0 + w_1x_1 + \cdots + w_dx_d \qquad (x_0 = 1) \\
  \Delta\left(y, \mathcal{F}_{\mathbf{w}}(\mathbf{x})
  \right)
  &= \frac12 \left(y -
    \mathcal{F}_{\mathbf{w}}(\mathbf{x}) \right)^2  \\
  &= \frac12 \left(y -
    \mathbf{x}^{\top}\mathbf{w} \right)^2  \\
\end{align*}

The objective is to find \(\mathbf{w}=\mathbf{w}_*\) in
order to

\begin{align*}
  \underset{\mathbf{w}}{\text{minimise}}
  &\quad \underset{y,\mathbf{x}\sim\mathcal{D}}{\mathbb{E}}
    \left[ \frac12 \left(y - \mathbf{x}^{\top}\mathbf{w}
    \right)^2 \right] \\
  \text{or,}\quad \underset{\mathbf{w}}{\text{minimise}}
  &\quad \frac12 (\mathbf{y}-X\mathbf{w})^{\top}
    (\mathbf{y}-X\mathbf{w}) \\
  \text{where,}\quad \mathbf{y}\equiv\begin{bmatrix}
    y_1 \\ \vdots \\ y_N
  \end{bmatrix} &\quad X \equiv \begin{bmatrix}
    \mathbf{x}_1^{\top} \\ \vdots \\
    \mathbf{x}_N^{\top}
  \end{bmatrix}
\end{align*}

The analytical solution yields,

\begin{align*}
  \mathbf{w}_* &= (X^{\top}X)^{-1}X^{\top}\mathbf{y}
\end{align*}

\subsection{Implementation}
\label{sec:org67103e8}

\subsubsection{In Spreadsheet}
\label{sec:org209d777}
This \href{https://docs.google.com/spreadsheets/d/1MrwsA75WUano\_aKpeOibALiCOumbdiQv-609A6fEC-c/edit?usp=sharing}{(Google Sheet)} will help understand and practice
computing the solution manually for the case in 2D.
\subsubsection{In Code}
\label{sec:org7da3954}
This \href{https://gist.github.com/bvraghav/4b81c850cd7f3c9784493a465ba592ca}{(Gist)} is a reference python implementation of the
analytical solution.


\section{Logistic Regression}
\label{sec:org1a5d668}

(Binary Classification)

\begin{align*}
  y \approx \widetilde{y} = \mathcal{F}_{\mathbf{w}}(\mathbf{x})
  &= \sigma(\mathbf{x}^{\top}\mathbf{w}) \\
  \Delta\left(y, \widetilde{y} \right)
  &= y\ln \widetilde{y} + (1-y)
    \ln (1-\widetilde{y}) \\
  \frac{\partial \Delta(y,\widetilde{y})} {\partial
  \mathbf{w}}
  &= (y-\widetilde{y})\mathbf{x}
\end{align*}

The objective is to find \(\mathbf{w}=\mathbf{w}_*\) in
order to

\begin{align*}
  \underset{\mathbf{w}}{\text{minimise}}
  &\quad \mathcal{L}(\mathbf{w}) = \underset{y,
    \mathbf{x} \sim \mathcal{D}}{\mathbb{E}}
    \left[ \Delta\left(y, \widetilde{y} \right) \right]
\end{align*}

There’s no analytical solution.  But using gradient
descent, we numerically hope to converge using
iterative update,

\begin{align*}
  \mathbf{w} &\gets \mathbf{w} -\lambda \frac {\partial
               \mathcal{L}} {\partial \mathbf{w}} \\
  &= \mathbf{w} -\lambda \, \underset{y, \mathbf{x}
    \sim \mathcal{D}} {\mathbb{E}} \left[
    (y-\widetilde{y}) \mathbf{x} \right] 
\end{align*}

\section{Support Vector Machine}
\label{sec:orgccaa349}

\begin{figure}[!h]
\LARGE
\centering
\def\svgwidth{0.8\linewidth}
\input{./svm.pdf_tex}
\caption{SVM Theory Illustration}
\end{figure}

\begin{enumerate}
\item Given a dataset \(\mathcal{D}\) with paired samples
\((y,\mathbf{x}); y\in\{+1,-1\}\) so that positive
samples are labeled \(y=+1\), and similarly negative
samples as \(y=-1\).
\item To evaluate for a simple case, let’s assume that the
positive and negative samples are “comfortably”
separable through \textbf{a hyperplane}.  In case of 2D
data \((\mathbf{x}\in\mathbb{R}^2)\), it would follow
from the assumption that there exists a straight
line with a finite margin, called \textbf{gutter space}
such that,
\begin{enumerate}
\item There are no samples in the gutter space;
\item Positive samples lie on one side of the
hyperplane; and
\item Negative samples lie on the other side.
\end{enumerate}
\item Our aim is to find the straight line that maximises
the gutter space.
\item Let the separating hyperplane (straight line in case
of 2D data) be given as,

\begin{align}
  \mathbf{w}\cdot\mathbf{x} + b = 0
\end{align}

Geometrically speaking, \(\mathbf{w}\) is a vector
normal to the separating hyperplane.  And the unit
vector in the same direction is given as
\(\mathbf{w}/\|\mathbf{w}\|_2\).  Where
\(\|\mathbf{w}\|_2\) is called the Frobenius Norm and
\(\|\mathbf{w}\|_2^2 = w_1^2+\cdots+w_d^2\).  This is
the same as the understanding of “magnitude” of the
vector in Euclidean space.

\item The hyperplane separates the space such that \\[0pt]
One side of it satisfies
\(\mathbf{w}\cdot\mathbf{x}+b < 0\); and \\[0pt]
The other side satisfies
\(\mathbf{w}\cdot\mathbf{x}+b > 0\).

\item From the separability assumption, it follows, \\[0pt]
\begin{align*}
  \mathbf{w}\cdot\mathbf{x}+b < 0 &\quad\forall y=-1 \\
  \mathbf{w}\cdot\mathbf{x}+b > 0 &\quad\forall y=+1
\end{align*}

\item From the margin assumption, without loss of
generality, it follows that \\[0pt]
\begin{align*}
   \mathbf{w}\cdot\mathbf{x}+b \leqslant -1 &\quad
   \forall y=-1 \\
   \mathbf{w}\cdot\mathbf{x}+b \geqslant 1 &\quad
   \forall y=+1
\end{align*}

\item In other words
\begin{align}
  y(\mathbf{w}\cdot\mathbf{x}+b) \geqslant 1
\end{align}

\item For the points on the margin, denoted as
\(\mathbf{x}_{+}, \mathbf{x}_{-}\) in the adjoining
image,

\begin{align}
  \notag\mathbf{w}\cdot\mathbf{x}_{+} + b &= 1 \\
  \notag\mathbf{w}\cdot\mathbf{x}_{-} + b &= -1 \\
  \mathbf{w}\cdot(\mathbf{x}_{+}-\mathbf{x}_{-}) &= 2
\end{align}

\item The gutter width \(\gamma\) is given as the
projection of vector \(\mathbf{x}_{+} -
    \mathbf{x}_{-}\) along the normal to the hyperplane.
Or,

\begin{align}
  \notag
  \gamma &= \frac{\mathbf{w}}{\|\mathbf{w}\|_2} \cdot
           (\mathbf{x}_{+}-\mathbf{x}_{-}) \\
  \notag &= \frac{\mathbf{w}\cdot
           (\mathbf{x}_{+}-\mathbf{x}_{-})}
           {\|\mathbf{w}\|_2} \\
  \gamma &= \frac{2}{\|\mathbf{w}\|_2}
\end{align}

Our aim is to maximise the gutter width \(\gamma\),
which would be the same as minimising \(1/\gamma\),
or \(1/\gamma^{2}\), or \(4/\gamma^{2} =
    \|\mathbf{w}\|_{2}^{2}\).
\end{enumerate}


\subsection{Training}
\label{sec:org9657c8d}

Formally speaking, we need to find the parameters
\(\mathbf{w},b\) in order to
\begin{align*}
  \text{minimise} &\quad\|\mathbf{w}\|_2^2 \\
  \text{such that,} &\quad y(\mathbf{w}\cdot\mathbf{x}
                      + b) \geqslant 1
\end{align*}

\subsection{Inference}
\label{sec:orgcc94428}

For all unseen points, \(\mathbf{x}\), the estimated
label \(\widehat{y}\) is given as,

\begin{align}
  \widehat{y} &= \mathrm{signum}(\mathbf{w}\cdot\mathbf{x}+b)
\end{align}

\subsection{Implementation}
\label{sec:org7cb663d}

Check out \href{https://gist.github.com/bvraghav/7d413048aaea04912a1e3d8872c0c8c4}{this gist}
\end{document}
